\documentclass[a4paper]{scrartcl}
\usepackage[]{csquotes} 
\usepackage[backend=biber, style=ieee, url=false, eprint=false, isbn=false]{biblatex} 
\addbibresource{top.bib}
\usepackage[]{graphicx} 
\graphicspath{{figures/}}
\usepackage[]{amsmath,amssymb} 
\usepackage[]{mathtools} 
\usepackage[]{unicode-math-xetex}
\usepackage[french]{babel} 
\usepackage[]{cleveref} 

% Glossary
\usepackage[acronym, postdot, toc]{glossaries-extra}
\setabbreviationstyle[acronym]{long-short}
\loadglsentries[acronym]{acronyms}
\makenoidxglossaries

\title{RO06 - Tournée de véhicule sélective}
\subtitle{Team Orienteering Problem} 
\author{Pascal Quach, Antoine Marquis}
\date{\today}

\begin{document}

\maketitle

Le problème de la "course d'orientation"\textemdash \gls{top} \textemdash est
un problème de tournées de véhicules sélectives.

Nous rappelons en \cref{sec:top-description} la description du problème. En
\cref{sec:méthodes-exactes}, nous citons diverses méthodes exactes, et en
\cref{sec:heuristiques} quelques heuristiques. En \cref{sec:implementation},
nous proposons diverses implémentations et les testons sur des instances
disponibles en ligne~\cite{cib_test_instances,chao_1993,chao.etal_feb1996,tsiligirides_sep1984}.

\section{Description du problème}%
\label{sec:top-description}

Soit une flotte de $m$ véhicules, auxquels est associé un temps de parcours
maximal $T_{\max}$, dont le but est de visiter des clients parmi les $n$
disponibles, en empruntant un itinéraire $r$ sans redondance. Les véhicules
sont associés deux dépôts spéciaux, le dépôt de départ, et celui d'arrivée.

Chaque client $i$ est associé à un montant $p_i$ correspondant au profit
pouvant être récolté une et une seule fois par un véhicule.

L'objectif est de fournir un ensemble d'itinéraires à emprunter pour les $m$
véhicules de telle sorte que le profit soit maximisé.

\section{Méthodes exactes}%
\label{sec:méthodes-exactes}

\begin{enumerate}
	\item \cite{butt.ryan_apr1999}
	\item \cite{boussier.etal_sep2007} 
	\item \cite{poggi.etal_2010}
	\item \cite{dang.etal_may2013}
	\item \cite{keshtkaran.etal_jan2016}
\end{enumerate}

\section{Heuristiques}%
\label{sec:heuristiques}

\section{Implémentation}
\label{sec:implementation}

\printnoidxglossary[type=acronym]
\printbibliography

\end{document}
